Temps 1 : 
	\subsection{Exposé la problématique}  
	\begin{enumerate}
		\item Il faut que classe mes photos ?  Comment faire ?
	pOur certaines c'est facile, ...
	Pour d'autre pas évident....



	Qu'est ce qu'on peut trouver dans une photo ? 
	les données EXIFs 

	Site ou on voit la géolocalisation de la photo
	https://tool.geoimgr.com/
	ou 
	FindEXIF.com 



	Avec un lien quelconques vers une activité données EXIF.


	Cherche sur Magistère un lien vers une activité sur les données EXIF.


    Conclusion : Il faut trouver un moyen pour lire et classer toutes ces données exifs. Automatiser tout ça et pouvoir le réutiliser dans un autre contexte.



Maxi photo de python


Pas à pas ,
On ouvre un editeur
on import un module
sur une photo

dans la console 
	on import un module
	on fait un __dir__ du module
	puis un help d'une fonction.
	1 fois on fait latitude


Voila on est prêt 
C'est chiant, on passe à un éditeur de texte.

On utlise sublim ou autre...
 	on retape import..
 	latitude ...
 	ça marche pas ....
 	On lui a pas dit d'afficher ...
 	Comment qu'on fait....
 	
 	....
 	Il vont chercher dans leur petit manuel
 	....
 	
 	....
 	La photo qui a pas de donnée? :( j'ai pas de donnée.

 	Est qu'on la traite ? Non Il ne pas les prendre ....
 	Comment qu'on fait :
 	Pseudo-code 
	 	lat <- latitude(img)
	 	Si lat n'existe pas  
	 		dire pas de chance
	 	Sinon 
	 		dire on va verifier qu'on est à mayotte



    Est-ce qu'on est à Mayotte ?

    Refactorisation .

   Lettre sur un moteur de recherhce .. pour vous aider.

   Mettre un minteur pour compter le nombre de photos de mayotte.

   Une solution parmis tant d'autres faire une liste.

   -------------------------
   -------------------------
PAUSE DE 10 min
   _________________________
   _________________________

On les affiche sur une carte 

..
TOPO Open street map ....
Guillaume parle pendant 0 min. 
..


Comment on va faire.

pip install ....



foto{}.jpg".format(i)

